%%%%%%%%%%%%%%%%%%%%%%%%%%%%%%%%%%%%%%%%%
% Beamer Presentation
% LaTeX Template
% Version 1.0 (10/11/12)
%
% This template has been downloaded from:
% http://www.LaTeXTemplates.com
%
% License:
% CC BY-NC-SA 3.0 (http://creativecommons.org/licenses/by-nc-sa/3.0/)
%
%%%%%%%%%%%%%%%%%%%%%%%%%%%%%%%%%%%%%%%%%

%----------------------------------------------------------------------------------------
%	PACKAGES AND THEMES
%----------------------------------------------------------------------------------------

\documentclass[usenames,dvipsnames,aspectratio=169,serif]{beamer}
\definecolor{cyanprocess}{rgb}{0.0, 0.72, 0.92}
\usecolortheme[named=cyanprocess]{structure}
\usepackage{xcolor}

\usepackage[T1]{fontenc}
\usepackage{fontawesome5}
\usepackage{concmath}
\usepackage{inconsolata}
\usepackage{draftwatermark}
\setbeamercolor{background canvas}{bg=}%transparent canvas
\SetWatermarkText{\ttfamily DRAFT}
\SetWatermarkScale{0.5}
\SetWatermarkAngle{30}
\mode<presentation> {

   % The Beamer class comes with a number of default slide themes
   % which change the colors and layouts of slides. Below this is a list
   % of all the themes, uncomment each in turn to see what they look like.

   %\usetheme{default}
   %\usetheme{AnnArbor}
   %\usetheme{Antibes}
   %\usetheme{Bergen}
   %\usetheme{Berkeley}
   %\usetheme{Berlin}
   %\usetheme{Boadilla}
   %\usetheme{CambridgeUS}
   %\usetheme{Copenhagen}
   %\usetheme{Darmstadt}
   %\usetheme{Dresden}
   %\usetheme{Frankfurt}
   %\usetheme{Goettingen}
   %\usetheme{Hannover}
   %\usetheme{Ilmenau}
   %\usetheme{JuanLesPins}
   %\usetheme{Luebeck}
   %\usetheme{Madrid}
   %\usetheme{Malmoe}
   %\usetheme{Marburg}
   %\usetheme{Montpellier}
   %\usetheme{PaloAlto}
   %\usetheme{Pittsburgh}
   %\usetheme{Rochester}
   %\usetheme{Singapore}
   %\usetheme{Szeged}
   \usetheme{Warsaw}

   % As well as themes, the Beamer class has a number of color themes
   % for any slide theme. Uncomment each of these in turn to see how it
   % changes the colors of your current slide theme.

   %\usecolortheme{albatross}
   %\usecolortheme{beaver}
   %\usecolortheme{beetle}
   %\usecolortheme{crane}
   %\usecolortheme{dolphin}
   %\usecolortheme{dove}
   %\usecolortheme{fly}
   %\usecolortheme{lily}
   %\usecolortheme{orchid}
   %\usecolortheme{rose}
   %\usecolortheme{seagull}
   %\usecolortheme{seahorse}
   %\usecolortheme{whale}
   %\usecolortheme{wolverine}

   %\setbeamertemplate{footline} % To remove the footer line in all slides uncomment this line
   %\setbeamertemplate{footline}[page number] % To replace the footer line in all slides with a simple slide count uncomment this line

   %\setbeamertemplate{navigation symbols}{} % To remove the navigation symbols from the bottom of all slides uncomment this line
   %\setbeamertemplate{frametitle}[default][colsep=-4bp,shadow=false,rounded=true]
   \setbeamertemplate{title page}[default][colsep=-0bp,rounded=true]
   \setbeamertemplate{blocks}[rounded][shadow=false]
   \setbeamertemplate{headline}[shadow=false]
   \setbeamertemplate{subsection in head}[shadow=false]
   \setbeamertemplate{section in head}[shadow=false]
   \setbeamertemplate{beamercolorbox}[shadow=false]

}
% \useoutertheme{smoothbars}
%
%
% \makeatletter
% \AtBeginDocument{
% \pgfdeclareverticalshading{beamer@barshade}{\the\paperwidth}{%
%          color(0ex)=(black);%
%          color(0.5ex)=(section in head/foot.bg);%
%          color(4ex)=(section in head/foot.bg)%
%        }
% }
% \makeatother


\usepackage{graphicx} % Allows including images
\usepackage{booktabs} % Allows the use of \toprule, \midrule and \bottomrule in tables

%\usepackage[unicode=true,
% bookmarks=true,bookmarksnumbered=true,bookmarksopen=true,bookmarksopenlevel=1,
% breaklinks=false,pdfborder={0 0 0},pdfborderstyle={},backref=false,colorlinks=true]
% {hyperref}
\hypersetup{
   urlcolor=cyanprocess,
   linkcolor=cyanprocess,
   filecolor=cyanprocess,
   citecolor=cyanprocess,
   pdftitle={Unmanned Traffic Management and Standardisation},
   pdfauthor={Sayandeep Purkayasth},
   pdfsubject={Unmanned Aviation},
   pdfkeywords={Unmanned Aircraft Systems, Unmanned Traffic Management, Standardisation},
}

\usepackage{tikz}
\usetikzlibrary{mindmap,trees,backgrounds}
\usetikzlibrary{shapes,arrows}
\usepackage{xypic}
\xyoption{curve}

%----------------------------------------------------------------------------------------
%	TITLE PAGE
%----------------------------------------------------------------------------------------

\title[UTM \& Stdn.]{Unmanned Traffic Management and Standardisation} % The short title appears at the bottom of every slide, the full title is only on the title page

\author{Sayandeep Purkayasth} % Your name
\institute[UAWGs] % Your institution as it will appear on the bottom of every slide, may be shorthand to save space
{
   \href{mailto:sayandeep@deepcyan.ai}{sayandeep@deepcyan.ai}  \\ % Your email address
   \medskip
   Unmanned Aviation Working Groups
   \footnote{\tiny \faLink \, https://groups.google.com/forum/\#!forum/utm-wg} % Your email address
   \footnote{\tiny \faEnvelope[regular] utm-wg@googlegroups.com} % Your email address
   \footnote{\tiny \faGit \, https://github.com/utm-working-group} % Your email address
}
\date{7 Nov 2020} % Date, can be changed to a custom date

\begin{document}

\begin{frame}
   \titlepage % Print the title page as the first slide
\end{frame}

\begin{frame}
   \frametitle{Overview} % Table of contents slide, comment this block out to remove it
   \tableofcontents % Throughout your presentation, if you choose to use \section{} and \subsection{} commands, these will automatically be printed on this slide as an overview of your presentation
\end{frame}

%----------------------------------------------------------------------------------------
%	PRESENTATION SLIDES
%----------------------------------------------------------------------------------------

%------------------------------------------------
\section{Introduction} % Sections can be created in order to organize your presentation into discrete blocks, all sections and subsections are automatically printed in the table of contents as an overview of the talk
%------------------------------------------------

\subsection{UA Working Groups} % A subsection can be created just before a set of slides with a common theme to further break down your presentation into chunks

\begin{frame}
   \frametitle{}

   \begin{figure}[tbh]
      \begin{centering}
         \begin{center}
            \small\tt
            \hfil
            \xymatrix{
               %& \txt{DFI} \ar[dr] & \txt{MoCA} \ar[d] & \cdots \ar[dl] & \\
               % & & \txt{UA Society} \ar[d] & & \\
               % & & \txt{Steering \\Committee} \ar[dl] \ar[dr] & & \\
               & & \txt{Working Groups} \ar@{-}@/_/[dll] \ar@{-}[dl] \ar@{-}[d] & \txt{Research Groups} \ar@{-}[d] \ar@{-}[dr] & \\
               *+{\txt{NPNT}} & *+[F--]{\txt{Remote ID}} & *+[F]{\txt{UTM}} & *+{\txt{Risk}} & *+{\txt{Deconfliction}} }
         \hfil \end{center}
      \par\end{centering}
      \caption{Unmanned Aviation Working and Research Groups}
   \end{figure}

\end{frame}

%------------------------------------------------

\subsection{Overview}

\begin{frame}
   \frametitle{Stakeholders}
   % MIND MAP

   \begin{columns}[t] % the "c" option specifies center vertical alignment
      \column{.45\textwidth} % column designated by a command
      \begin{tikzpicture}[scale=0.5,transform shape]
         \ttfamily
         \path[mindmap,concept color=cyanprocess,text=white]
         node[concept] {Unmanned Traffic Management}
         [clockwise from=0]
         child [concept color=RoyalBlue!50!cyanprocess] { node[concept] (c3) {UAS} }
         child [concept color=OliveGreen] { node[concept] (c1) {Operator} }
         child [concept color=Maroon] { node[concept] (c2) {Pilot} }
         child [concept color=black] { node[concept] (c5) {Manned Air Traffic Control} }
         child [concept color=YellowOrange] { node[concept] (c4) {Civil Aviation Authority} }
         child [concept color=Violet] { node[concept ] (c0) {Manufacturer} };
         \begin{pgfonlayer}{background}
            \draw [concept connection]  (c1) edge (c2)
            edge (c3)
            (c2) edge (c3);
         \end{pgfonlayer}
      \end{tikzpicture}
      \column{.45\textwidth}
      Contents split \newline into two lines
   \end{columns}
\end{frame}

%------------------------------------------------

\begin{frame}
   \frametitle{Need}
   \begin{columns}[t]
      \column{.45\textwidth}
      \begin{itemize}
         \item Increasing number of drones, flights, pilots
         \item Communication between stakeholders
         \item Enabling newer use cases
         \item Maintaining operational privacy, safety and security
            \begin{itemize}
               \item Situational awareness
               \item Separation
            \end{itemize}
      \end{itemize}
      \column{.45\textwidth}
      \begin{itemize}
         \item Deconfliction
            \begin{itemize}
               \item Other Unmanned air traffic
               \item Manned air traffic
            \end{itemize}
         \item Regulatory compliance
         \item Supporting safer flight planning
         \item Identification of risk factors for complex operations
      \end{itemize}
   \end{columns}

\end{frame}

%------------------------------------------------

\begin{frame}
   \frametitle{Services: Registration}
   \begin{itemize}
      \item UAS \\
         UIN Application, UAS Acquisition
      \item Manufacturer \\
         Profile Management

      \item Operator \\
         Profile \& Permission Management, UAOP Application

      \item Pilot \\
         Profile Management
   \end{itemize}
\end{frame}

%------------------------------------------------

\begin{frame}
   \frametitle{Services: Operational}
   \begin{columns}[t] % The "c" option specifies centered vertical alignment while the "t" option is used for top vertical alignment

      \column{.45\textwidth} % Left column and width
      \begin{itemize}
         \item Flight Planning
         \item Flight Awareness
         \item Communication \& Navigation
         \item Dynamic Airspace Density
         \item Discovery
      \end{itemize}

      \column{.45\textwidth} % Right column and width
      \begin{itemize}
         \item Log Management
         \item Weather
         \item Mapping
         \item Airspace authorisation
         \item Incident Reporting
      \end{itemize}
   \end{columns}
\end{frame}
%------------------------------------------------

\begin{frame}
   \frametitle{Functions: Operational (contd.)}
   \begin{columns}[t] % The "c" option specifies centered vertical alignment while the "t" option is used for top vertical alignment

      \column{.45\textwidth} % Left column and width
      \begin{itemize}
         \item Deconfliction
            \begin{itemize}
               \item Advisory \& Alert
               \item Strategic
               \item Tactical / Dynamic Reroute
            \end{itemize}
      \end{itemize}

      \column{.45\textwidth} % Right column and width
      \begin{itemize}
         \item Restrictions
         \item Conformance Monitoring
         \item Risk reduction
         \item Messaging
         \item Flight Notification
      \end{itemize}
   \end{columns}

\end{frame}

\subsection{Nomenclature}

\begin{frame}
   \frametitle{High level}
   \begin{block}{Block 1}
      Lorem ipsum dolor sit amet, consectetur adipiscing elit. Integer lectus nisl, ultricies in feugiat rutrum, porttitor sit amet augue. Aliquam ut tortor mauris. Sed volutpat ante purus, quis accumsan dolor.
   \end{block}

   \begin{block}{Block 2}
      Pellentesque sed tellus purus. Class aptent taciti sociosqu ad litora torquent per conubia nostra, per inceptos himenaeos. Vestibulum quis magna at risus dictum tempor eu vitae velit.
   \end{block}

   \begin{block}{Block 3}
      Suspendisse tincidunt sagittis gravida. Curabitur condimentum, enim sed venenatis rutrum, ipsum neque consectetur orci, sed blandit justo nisi ac lacus.
   \end{block}
\end{frame}

%------------------------------------------------

\begin{frame}
   \frametitle{High level (contd.)}
   \begin{columns}[t] % The "c" option specifies centered vertical alignment while the "t" option is used for top vertical alignment

      \column{.45\textwidth} % Left column and width
      \textbf{Heading}
      \begin{enumerate}
         \item Statement
         \item Explanation
         \item Example
      \end{enumerate}

      \column{.5\textwidth} % Right column and width
      Lorem ipsum dolor sit amet, consectetur adipiscing elit. Integer lectus nisl, ultricies in feugiat rutrum, porttitor sit amet augue. Aliquam ut tortor mauris. Sed volutpat ante purus, quis accumsan dolor.

   \end{columns}
\end{frame}

%------------------------------------------------
\section{Services}
%------------------------------------------------

\subsection{Registration}
\begin{frame}
   \frametitle{Registration}
   \framesubtitle{UAS}
\end{frame}

\begin{frame}
   \frametitle{Registration}
   \framesubtitle{Manufacturer}
\end{frame}

\begin{frame}
   \frametitle{Registration}
   \framesubtitle{Operator}
\end{frame}

\begin{frame}
   \frametitle{Registration}
   \framesubtitle{Pilot}
\end{frame}

%------------------------------------------------


\subsection{Operations}
\begin{frame}
   \frametitle{Flight Planning}
\end{frame}

\begin{frame}
   \frametitle{Flight Awareness}
\end{frame}

\begin{frame}
   \frametitle{Communication \& Navigation}
\end{frame}

\begin{frame}
   \frametitle{Dynamic Airspace Density}
\end{frame}

\begin{frame}
   \frametitle{Discovery}
\end{frame}

\begin{frame}
   \frametitle{Log Management}
\end{frame}

\begin{frame}
   \frametitle{Weather}
\end{frame}

\begin{frame}
   \frametitle{Mapping}
\end{frame}

\begin{frame}
   \frametitle{Airspace authorisation}
\end{frame}

\begin{frame}
   \frametitle{Incident Reporting}
\end{frame}


\begin{frame}
   \frametitle{Restrictions}
\end{frame}

\begin{frame}
   \frametitle{Conformance Monitoring}
\end{frame}

\begin{frame}
   \frametitle{Risk reduction}
\end{frame}

\begin{frame}
   \frametitle{Messaging}
\end{frame}

\begin{frame}
   \frametitle{Flight Notification}
\end{frame}


\subsection{Deconfliction}

\begin{frame}
   \frametitle{Deconfliction}
\end{frame}

\begin{frame}
   \frametitle{Deconfliction}
   \framesubtitle{Advisory \& Alert}
\end{frame}

\begin{frame}
   \frametitle{Deconfliction}
   \framesubtitle{Strategic}
\end{frame}

\begin{frame}
   \frametitle{Deconfliction}
   \framesubtitle{Tactical / Dynamic Reroute}
\end{frame}

%------------------------------------------------

\section{Standardisation}
\subsection{Standardisation}
\begin{frame}[fragile] % Need to use the fragile option when verbatim is used in the slide
   \frametitle{Standardisaton}
   \framesubtitle{Efforts}
   India and regulatory changes
   \begin{block}{NPNT}
      Improving standardisation and documentation
   \end{block}

   \begin{block}{DigitalSky\texttrademark}
      Recommendations for improvements and features
   \end{block}

\end{frame}

%------------------------------------------------

\subsection{Challenges}
\begin{frame}[fragile] % Need to use the fragile option when verbatim is used in the slide
   \frametitle{Standardisaton}
   \framesubtitle{Challenges}
   \begin{columns}[t]
      \column{.45\textwidth} % column designated by a command
      \begin{itemize}
         \item Rapidly evolving technology
            \begin{itemize}
               \item UAS internals
               \item Mobile Connectivity Infrastructure
               \item New commercial applications
            \end{itemize}
         \item Industry consensus difficult to achieve
            \begin{itemize}
                  \item Too many stakeholders
                  \item Diverse industries
            \end{itemize}
      \end{itemize}
      \column{.45\textwidth} % column designated by a command
      \begin{itemize}
         \item India still catching up on regulatory changes in EU/US, etc.
            \begin{itemize}
               \item Separation standards
               \item Remote Identification and tracking
            \end{itemize}
         \item Chicken before egg: Regulatory effort underway before industry has reached scale
         \item Reliable network connectivity
      \end{itemize}
   \end{columns}
\end{frame}

%------------------------------------------------

\begin{frame}[fragile] % Need to use the fragile option when verbatim is used in the slide
   \frametitle{Standardisaton}
   \framesubtitle{Challenges}
   \begin{columns}[t]
      \column{.45\textwidth} % column designated by a command
      \begin{itemize}
         \item Hello
      \end{itemize}
      \column{.45\textwidth} % column designated by a command
      \begin{itemize}
         \item No concerted effort on standardising the way operational practices or risk assessments are evaluated by regulatory bodies
      \end{itemize}
   \end{columns}
\end{frame}

%------------------------------------------------


\begin{frame}
   \frametitle{Figure}
   Uncomment the code on this slide to include your own image from the same directory as the template .TeX file.
   %\begin{figure}
   %\includegraphics[width=0.8\linewidth]{test}
   %\end{figure}
\end{frame}

%------------------------------------------------

\begin{frame}[fragile] % Need to use the fragile option when verbatim is used in the slide
   \frametitle{Citation}
   An example of the \verb|\cite| command to cite within the presentation:\\~

   This statement requires citation \cite{FAA-ConOps-v2}.
   This statement requires citation \cite{EASA-UAS-ATM-Assessment-2018}.
   This statement requires citation \cite{RPASGM2020}.
   This statement requires citation \cite{FAA-NPRM-2019-1100}.
\end{frame}

%------------------------------------------------

\begin{frame}
   \frametitle{Figure}
   Uncomment the code on this slide to include your own image from the same directory as the template .TeX file.
   %\begin{figure}
   %\includegraphics[width=0.8\linewidth]{test}
   %\end{figure}
\end{frame}

%------------------------------------------------

\begin{frame}[t, allowframebreaks]
   \frametitle{References}
   \bibliographystyle{amsalpha}
   \bibliography{../../research/Bibliography}
\end{frame}

%------------------------------------------------

\begin{frame}
   \Huge{\centerline{The End}}
\end{frame}

%----------------------------------------------------------------------------------------

\end{document}

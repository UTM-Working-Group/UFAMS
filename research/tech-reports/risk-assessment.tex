\documentclass{ua_wgs_base}

\drafttrue
\doctype{Technical Report}
\doctypeinitials{TR}
\docid{1}
\workinggroup{UA Risk Research Group}
\workinggroupinitials{UARRG}
\doctitle{Risk Assessment}

\hypersetup{pdftitle={Risk Assessment for Unmanned Aircraft Systems},
 pdfauthor={\printworkinggroup},
 pdfsubject={\printdoctitle},
 pdfkeywords={Unmanned Aircraft Systems, Risk Assessment},
 pdfpagelayout=OneColumn, pdfnewwindow=true, pdfstartview=XYZ, plainpages=false}

\printdraftmark

\begin{document}

\title{Risk Assessment for Unmanned Aircraft Systems}

\author{\printworkinggroup}

\maketitle
\cleardoublepage{}
\begin{abstract}
This paper describes a risk assessment study of Unmanned Aircraft
Systems that aims to understand the risks, underlying causes, consequences
and mitigations associated with their operation. 

Risk Categorisation; Simulations (maybe in a subsequent study); 

Keywords: Unmanned Aviation, Risk Assessment, Hazard Identification,
Mitigation Strategies
\end{abstract}

\chapter*{UA Risk Research Group\label{sec:wg}}

\addcontentsline{toc}{section}{\nameref{sec:wg}}

This working group meets almost weekly to discuss the roles, responsibilities
and activities to be performed by a Unmanned Traffic Management Service
Provider (UTM-SP)\nomenclature{DSP}{DigitalSky Service Provider; used interchangeably with UTM-SP}\nomenclature{UTM}{Unmanned Traffic Management; an activity similar to Air Traffic Management for manned flights}\nomenclature{UTM-SP}{UTM Service Provider; an organisation providing UTM services, including but not limited to Remote Identification and Tracking of UASs; used interchangeably with DSP}
in the context of Indian drone industry. It also aims to document
the ways in which a UTM-SP interacts with other stakeholders in the
industry, namely, manufacturers, operators, pilots, UFII, and so forth.
The documentation would be put through as recommendations to DGCA\nomenclature{DGCA}{Directorate General of Civil Aviation, MoCA},
MoCA,\nomenclature{MoCA}{Ministry of Civil Aviation, Government of India}
and other relevant parties as it matures.

\begin{figure}[tbh]
\begin{centering}
\begin{center}
\small\tt
\hfil 
\xymatrix{
 % & \txt{DFI} \ar[dr] & \txt{MoCA} \ar[d] & \cdots \ar[dl] & \\ 
% & & \txt{UA Society} \ar[d] & & \\
& & \txt{Steering \\Committee} \ar[dl] \ar[dr] & & \\
& \txt{Working Groups} \ar[dl] \ar[d] & & \txt{Research Groups} \ar[dl] \ar[d] \ar[dr] & \\
\cdots \ar[r] & {\txt{UTM}} \ar[l] \ar[r] & *+[F]{\txt{Risk}} \ar[l] \ar[r] & \txt{Deconfliction} \ar[l] \ar[r] & \cdots \ar[l] }
\hfil \end{center}
\par\end{centering}
\caption{The Risk Research Group and others}
\end{figure}


\paragraph*{Membership}
\begin{center}
\begin{tabular}{|c|c|}
\hline 
iSpirt, India & Asteria, India\tabularnewline
Algopixel Technologies, India & HexCod, India\tabularnewline
Deepcyan Software, India & Curl Analytics, India\tabularnewline
Skylark Drones, India & Avianco, India\tabularnewline
Drone Federation of India, India & IdeaForge, India\tabularnewline
UniFly, Belgium & AUS, India\tabularnewline
Drone Aerospace, India & Omnipresent Tech, India\tabularnewline
QTPI, India & IIT Bombay, India\tabularnewline
IISc, India & \tabularnewline
\hline 
\end{tabular}
\par\end{center}

\paragraph*{Topics}

Standardisation, regulation, policy making in unmanned aviation. 

\paragraph*{Meetings}
\begin{center}
\begin{tabular}{|c|c|c|}
\hline 
Working Groups & \href{https://meet.google.com/urz-pekp-rwr}{Meet} & Fridays 1900-2000 IST\tabularnewline
Research Groups & \href{https://join.skype.com/RkNeUB8wNvzx}{Skype} & Wednesdays 1900-2000 IST\tabularnewline
\hline 
\end{tabular}
\par\end{center}

\paragraph*{Links}
\begin{itemize}
\item \texttt{https://groups.google.com/forum/\#!forum/utm-wg}
\item \texttt{utm-wg@googlegroups.com} (members only\footnote{Visit \texttt{https://groups.google.com/forum/\#!contactowner/utm-wg}
for membership.})
\end{itemize}

\chapter*{Authors\label{sec:authors}}

\addcontentsline{toc}{section}{\nameref{sec:authors}}
\begin{center}
\begin{tabular}{|c|c|c|}
\hline 
\textbf{Name} & \textbf{Email} & \textbf{Organisation}\tabularnewline
\hline
Sayandeep Purkayasth\footnotemark & \texttt{\href{mailto:sayandeep@deepcyan.ai}{sayandeep@deepcyan.ai}} & Deepcyan Software, India\tabularnewline
Manish Shukla & \texttt{\href{mailto:manish.shukla393@gmail.com}{manish.shukla393@gmail.com}} & Zeta\tabularnewline
Apurva Joshi & \texttt{\href{mailto:zapurva@gmail.com}{zapurva@gmail.com}} & IIT Bombay\tabularnewline
 &  & \tabularnewline
\hline 
\end{tabular}
\par\end{center}

\footnotetext{Authors for correspondence.}

\settowidth{\nomlabelwidth}{UTM-SP}
\printnomenclature{}\cleardoublepage{}

\chapter{Introduction}

Here is the text text text text text text text text text text text
text text text text text.

\begin{figure}[tbh]
\begin{center}
  \begin{tikzpicture}[auto,
    block_center/.style ={rectangle, draw=black, thick, fill=white,
      text width=15	em, text centered,
      minimum height=4em, font=\ttfamily},
    block_center_hi/.style ={rectangle, draw=black, thick, fill=cyanprocess,
      text width=15	em, text centered,
      minimum height=4em, font=\ttfamily},
    block_center_free/.style ={rectangle,
      text width=8em, text badly ragged,
      minimum height=4em, font=\ttfamily},
   line/.style ={draw, thick, -latex', shorten >=0pt}]
    % outlining the flowchart using the PGF/TikZ matrix funtion
    \matrix [column sep=5mm,row sep=3mm] {
      % row 1
      \node [block_center_hi] (draft1) {Document well-known Standard Scenarios \& Use-cases}; & \node[block_center](draft1a){Create Infographic on Operational categorisation}; \\
      % row 2
      \node [block_center_hi] (draft2) {Document new Standard Scenarios \& Use-cases}; & \node[block_center](publish1){Industry session (wide) on Standard Scenarios, Use-cases \& Infographic}; \\
      % row 3
      \node [block_center] (draft3) {Draft TR Ch. 3 addressing requirements for studies}; & \node [block_center] (draft3a) {Create Infographics for TR Ch. 3}; \\
      % row 4
       & \node[block_center](publish2){Industry sessions on updates to Standard Scenarios \& Use-cases (wide) and feedback on TR Ch. 3 (narrow)};\\
      % row 5
      \node [block_center] (draft4) {Redraft TR Ch. 3}; &  \\
      % row 6
       & \node [block_center] (consensus1) {Industry session (wide) on Feedback from \& Consensus on TR Ch. 3}; \\
      % row 7
	  \node [block_center] (publish) {Finalise TR Ch. 3 \& Paper and publish}; & \\
      % row 8
    };% end matrix
    % connecting nodes with paths
    \begin{scope}[every path/.style=line]
      % paths for enrollemnt rows
      \path (draft1) -- (draft2);
      \path (draft1) -- (publish1);
      \path (draft2) -- (publish1);
      \path (draft1a) -- (publish1);
      \path (draft4) -- (consensus1);
      \path (draft3) -- (publish2);
      \path (publish2) -- (draft4);
      \path (draft3) -- (draft4);
      \path (draft3a) -- (publish2);
      \path (consensus1) -- (publish);
    \end{scope}
  \end{tikzpicture}
\end{center}

\caption{Workflow for Risk Assessment publications}
\end{figure}

\section{Scope}

Operations are typically classified into Open, Specific and Restricted.
In this report, we discuss the state of art in the risk assessment
field and attempt to devise a risk assessment strategy to address
the emerging operational requirements in the UA industry.

\section{Previous Work}

Joint Authorities on Rulemaking for Unmanned Systems (JARUS) Specific
Operations Risk Assessment (SORA) \cite{JARUS-SORA/JAR-DEL-WG6-D.04}
is a detailed study on the Operational Risks associated with UA operations.
It has been endorsed by the European Aviation Safety Agency (EASA).
Other risk assessment methodologies include $\ldots$

\subsection{JARUS SORA}

The SORA methodology categorises any UAS operation from ground (Ground
Risk Category - GRC) and air risk (Air Risk Category - ARC) perspectives.
These also determine a Specific Assurance and Integrity Levels (SAIL)
for ground and air, which represent the confidence that the operations
will stay under control within the intended operation. Based on the
assigned risk categories, mitigations are suggested to reduce SAIL.
The ultimate step of risk assessment is recommendation of Operational
Safety Objectives (OSO) to be met in accordance to SAIL. The SORA
process also details several operational scenarios that are intended
as a template for competent authorities and operators. 

Ultimately, however, the SORA process remains a qualitative assessment
strategy, and is does not contain prescriptive requirements, and aims
only to guide reduction of risk to acceptable levels.

\subsection{NASA UTM}

\begin{lyxgreyedout}
@Sayandeep: Include from https://docs.google.com/document/d/1eHDIJ4yoV9JgGWHc02FygwE4SounAoLkCo1Ykyhjjro/edit?usp=sharing
\&

NASA: https://drive.google.com/file/d/1OzQeRPdMM7s51fZJxoNRIl\_G-0QeHiTw/view?usp=sharing
pp48,79-80f%
\end{lyxgreyedout}

\begin{lyxgreyedout}
@Apurva%
\end{lyxgreyedout}

NASA\textquoteright s vision towards enabling \textquotedblleft in-time
and predictive safety assurance capabilities\textquotedblright{} for
highly autonomous operations of UASs in low altitude urban environments
has been outlined in \cite{NASA-TM-2020-220440}. The report proposes
a system architecture based on the \textquotedblleft Monitor-Assess-Mitigate\textquotedblright{}
paradigm and provides guiding principles for development. It also
discusses the span of information requirements to achieve safety assurance.
Further, it discusses the approaches taken towards uncertainty management,
models for information exchange and protocols. The results of various
research activities related to risk management have been outlined
in the appendices.

\subsection{U-Space EU ConOps v2.0}

\begin{lyxgreyedout}
Include from https://docs.google.com/document/u/1/d/1hLLxnDEfu928wVVeFjSoQvQnl\_pe8v3WvbCOA\_FmcQE/edit%
\end{lyxgreyedout}

This essentially follows the SORA process. Notable exceptions are:
\begin{enumerate}
\item $\ldots$
\end{enumerate}

\subsubsection{EASA}

See \cite{EASA-Op-2001-2018,EASA-Op-2005-2019,EASA-Ops-Categ,EASA-UAS-ATM-Assessment-2018}

Categorisation:
\begin{enumerate}
\item Open
\begin{enumerate}
\item A1
\item A2
\item A3
\end{enumerate}
\item Specific
\begin{enumerate}
\item Standard Scenarios
\item Non-standard Scenarios
\end{enumerate}
\end{enumerate}

\subsection{FAA ConOps v2.0}

\cite{FAA-ConOps-v2}

\subsection{BVLOS Indian efforts}

\subsubsection{@Dunzo Consortium}

\subsection{Scope of this work}

\section{Risk Assessment Strategies}

\cleardoublepage{}

\chapter{Operational Scenarios and Use Cases}

Operational scenarios and use cases considered in previous listerature
are listed in tables \ref{tab:Operational-Scenarios-old} and \ref{tab:Operational-Use-cases-old};
those not considered in previous literature are listed in tables \ref{tab:new-scenarios}
and \ref{tab:new-use-cases}, respectively. Each of the above along
with those in previous literature are detailed in the following sections.

\begin{table}[tbh]
\small
\ttfamily
\begin{centering}
\begin{tabular}{|l|l|l|l|}
\hline 
\textbf{\#} & \textbf{Cf.} &  & \tabularnewline
\hline 
\hline 
S01 & SORA/STS01 & VLoS & \tabularnewline
\hline 
S02 & SORA/STS02 & BVLoS & \tabularnewline
\hline 
S03 & FAA/V2-1 & Nominal UTM Operations in Uncontrolled  & \tabularnewline
 & & and Controlled Airspace & \tabularnewline
\hline 
S04 & FAA/V2-2 & UVRs and Associated Operational Impacts & \tabularnewline
\hline 
S05 & FAA/V2-3 & Interactions between UAS and Manned & \tabularnewline
 & &  Aircraft at Low Altitudes & \tabularnewline
\hline 
S06 & FAA/V2-4 & Use of UTM to Remotely Identify UAS & \tabularnewline
\hline 
S07 & FAA/V2-5 & Federal Public Safety Request for UTM & \tabularnewline
 & & Information & \tabularnewline
\hline 
 & EASA/STS-01 &  & $\equiv$S01\tabularnewline
\hline 
 & EASA/STS-02 &  & $\equiv$S02\tabularnewline
\hline 
\end{tabular}
\par\end{centering}
\caption{Operational Scenarios in previous literature\label{tab:Operational-Scenarios-old}}
\end{table}
\begin{sidewaystable}
\small
\ttfamily
\begin{centering}
\begin{tabular}{|l|l|}
\hline 
\textbf{\#} & \textbf{Description}\tabularnewline
\hline 
\hline 
TCL1-1 & Two VLOS Operations with Voluntary Use of UTM for Coordination\tabularnewline
\hline 
TCL2-1 & One BVLOS Operation, One VLOS Operation with Voluntary UTM Participation
for Coordination\tabularnewline
\hline 
TCL2-2 & Two BVLOS Operations near an Airport in Uncontrolled Airspace\tabularnewline
\hline 
TCL2-3 & Priority Operation \textendash{} Emergency Medical Aircraft in Uncontrolled
Airspace\tabularnewline
\hline 
TCL2-4 & BVLOS Operation Conformance Violation from Uncontrolled Airspace into
Class D Airspace\tabularnewline
\hline 
TCL3-1 & One-Way BVLOS Flight, Separate Landing/Take-Off Locations\tabularnewline
\hline 
TCL3-2 & Negotiation versus Prioritization between Operators Due to Dynamic
Restriction\tabularnewline
\hline 
TCL3-3 & UAS Interaction with Manned Aircraft in Low-Altitude Uncontrolled
Airspace\tabularnewline
\hline 
TCL3-4 & BVLOS Operation Lost-Link Event\tabularnewline
\hline 
TCL3-5 & High Density UTM Operations in Uncontrolled Airspace\tabularnewline
\hline 
TCL3-6 & Last-Mile Rural Deliveries in Uncontrolled Airspace under the Mode
C Veil\tabularnewline
\hline 
TCL3-7 & UAS Operator Loss of Performance Capabilities in Uncontrolled Airspace\tabularnewline
\hline 
TCL4-1 & BVLOS UTM Operation within UAS Facility Maps\tabularnewline
\hline 
TCL4-2 & Historical UTM Information Queries by Authorized Entities\tabularnewline
\hline 
TCL4-3 & UAS Urgency/Distress Condition with Alternate Landing and UTM Coordination\tabularnewline
\hline 
TCL4-4 & UAS Volume Reservation in Controlled Airspace\tabularnewline
\hline 
TCL4-5 & Report to FAA due to UAS Flight Incident\tabularnewline
\hline 
 & \tabularnewline
\hline 
\end{tabular}
\par\end{centering}
\caption{Operational Use cases in previous literature\label{tab:Operational-Use-cases-old}}
\end{sidewaystable}

\begin{center}
\begin{table}[tbh]
\small
\ttfamily
\begin{centering}
\begin{tabular}{|l|l|}
\hline 
\textbf{\#} & \textbf{Description}\tabularnewline
\hline 
\hline 
S01 & TBD\tabularnewline
\hline 
 & \tabularnewline
\hline 
 & \tabularnewline
\hline 
 & \tabularnewline
\hline 
 & \tabularnewline
\hline 
 & \tabularnewline
\hline 
\end{tabular}
\par\end{centering}
\caption{New Scenarios\label{tab:new-scenarios}}
\end{table}
\par\end{center}

\begin{center}
\begin{table}[tbh]
\small
\ttfamily
\centering{}%
\begin{tabular}{|l|l|}
\hline 
\textbf{\#} & \textbf{Description}\tabularnewline
\hline 
\hline 
UC01 & Night-time VLoS\tabularnewline
\hline 
UC02 & Night-time BVLoS\tabularnewline
\hline 
UC03 & Long range Fixed Wing Survey\tabularnewline
\hline 
UC04 & High EM Interference areas\tabularnewline
\hline 
UC05 & OOP\tabularnewline
\hline 
UC06 & Agricultural 001\tabularnewline
\hline 
UC07 & Agricultural 002\tabularnewline
\hline 
 & \tabularnewline
\hline 
 & \tabularnewline
\hline 
 & \tabularnewline
\hline 
\end{tabular}\caption{New Use cases\label{tab:new-use-cases}}
\end{table}
\par\end{center}

\section{Scenarios}

\paragraph{EASA Overview}

To limit the administrative burden for both UAS operators and the
competent authorities, a system of standard scenarios has been proposed.
A standard scenario involves a pre-established risk assessment and
includes mitigation measures. It may be followed by a declaration
submitted by the UAS operator (if the implementation of the mitigation
measures is considered to be simple), or by an authorisation issued
by the competent authority (when the implementation of the mitigation
measures is considered to be more complex). Furthermore, an optional
light UAS operator certificate (LUC) has been proposed, which allows
the competent authority to issue privileges to UAS operators. This
implies a significant investment from the operator\textquoteright s
side, which should yield benefits in the medium/long term. Indeed,
the LUC privileges can ultimately allow an operator to approve their
own operations.

\subparagraph{Approach}

STSs will be developed only for UAS operations in the \textquoteleft specific\textquoteright{}
category with a low risk (i.e. with a specific assurance and integrity
level (SAIL), as defined in SORA, not greater than 2).

\subsection{JARUS-STS-01}

This operational scenario detailed in \cite{JARUS-SORA/STS-01} covers
a UAS operation performed in the Specific category (Category B) as
follows.
\begin{itemize}
\item UA max characteristic dimension of 3m and typical kinetic energy up
to 34 kJ.
\item BVLOS w/ or w/o Visual Observer (VO)
\item Overflown area: Sparsely populated
\item Flight height limit: 150m or 500ft (above ground)
\item Airspace: uncontrolled (F or G)
\end{itemize}
\cite{JARUS-SORA/JAR-DEL-WG6-D.04} categorises such an operation
as follows
\begin{enumerate}
\item Final Ground Risk Class (GRC): \textbf{3}
\item Final Air Risk Class (ARC): \textbf{ARC-b}
\item SAIL: \textbf{II}
\end{enumerate}
and proceeds to detail the operational mitigations, operator provisions,
and technical provisions for such an operation.

\paragraph{EASA Variant}

STS-01: visual line of sight (VLOS) operations at a maximum height
of 120 m, at a ground speed of less than 5 m/s in the case of untethered
UA, over controlled ground areas that can be in populated (e.g. urban)
environments, using UAS with maximum take-off masses (MTOMs) of up
to 25 kg.

This limitation is a little more conservative than the in-service
experience of some MSs where UAS operations similar to STS-01 are
allowed up to a height of 150 m (500 ft). In STS-01, a 30 m margin
above the maximum height has been considered for use in abnormal situations.

For the ground risk buffer, \textquoteleft low\textquoteright{} robustness
is considered sufficient in UAS operations with a low intrinsic ground
risk. In this case, SORA indicates the 1:1 rule to select the minimum
horizontal distance. However, the 1:1 rule may lead to a buffer size
such that the size of the controlled ground area might be impractical
\textemdash{} in most cases, in a populated environment. Therefore,
the decision was made to propose more suitable values considering
the following elements:

To better ensure that the UA flight can be terminated without exceeding
the ground risk buffer, UAS operations under this STS are limited
to
\begin{enumerate}
\item Any configuration except fixed-wing UA. With this limitation, UAS
operations at low speed can be better ensured, and the likelihood
of the UA gliding a distance great enough for it to fall outside the
controlled ground area is minimised.
\item The ground speed in normal operation is limited to 5 m/s (which must
be set in the UAS) so that the controllability of the UA is increased.
\item For UA with MTOMs of up to 10 kg, in-service experience from MSs has
been considered. In particular, the main reference is French scenario
S-3, where a safety area is calculated assuming a ballistic fall once
the flight termination system is triggered, and therefore, the size
of that area is dependent on the flight height and speed of the UA. 
\item For UA with MTOMs above 10 kg, in-service experience is also considered,
but since this experience is more limited, a more conservative approach
is followed. In this case, the values considered were half of those
derived from the 1:1 rule, except that a minimum of 20 m is considered
in the case of a height of up to 30 m (thus, the values for UA above
10 kg are at least double those for the ones below 10 kg);
\end{enumerate}
Rules on remote pilot competency 

Contingency and emergency procedures to be defined in OM by operator

Emergency response plan must be devised 

Technical requirements in STS-01:

\subsection{JARUS-STS-02}

This operational scenario detailed in \cite{JARUS-SORA/STS-02} covers
a UAS operation performed in the Specific category (Category B) as
follows.
\begin{itemize}
\item UA max characteristic dimension of 3m and typical kinetic energy up
to 34 kJ.
\item VLOS
\item Overflown area: Sparsely populated
\item Flight height limit: reserved airspace for operation
\item Airspace: reserved airspace for operation
\end{itemize}
\cite{JARUS-SORA/JAR-DEL-WG6-D.04} categorises such an operation
as follows
\begin{enumerate}
\item Final Ground Risk Class (GRC): \textbf{3}
\item Final Air Risk Class (ARC): \textbf{ARC-a}
\item SAIL: \textbf{II}
\end{enumerate}
and proceeds to detail the operational mitigations, operator provisions,
and technical provisions for such an operation.

\paragraph{EASA Variant}

STS-02: beyond visual line of sight (BVLOS) operations with the UA
at not more than 2 km from the remote pilot, if visual observers (VOs)
are used, at a maximum height of 120 m, over controlled ground areas
in sparsely populated environments, using UA with MTOMs of up to 25
kg.

\subparagraph{Description}

STS-02 refers to a UAS operation with an increased intrinsic risk
compared with STS-01 because it allows BVLOS operations. The launch
and recovery of the UAS is, in any case, required to be performed
in VLOS. The main mitigation means is provided by VOs who assist the
remote pilot in scanning the airspace for the presence of other airspace
users. 

STS-02 should have the same height limitation as for STS-01

Air risk: mitigations for BVLOS:
\begin{enumerate}
\item a minimum visibility of 5 km is proposed to ensure the detection of
any potential hazard in the air. This was proposed by JARUS in the
frame of the SORA development and is also established in the regulations
covering UAS operations in some states. It is up to the UAS operator
to establish how to obtain the most suitable information to comply
with this requirement. Guidance material (GM) will be provided indicating
that operators may use meteorological reports to obtain visibility
data.
\item someone is always required to scan the airspace to detect any potential
hazards in the air. If no VO is used, then the scanning must be conducted
by the remote pilot. From experience in some states, having the UA
at not more than 1 km from the remote pilot (in combination with the
120 m height limitation) is considered a suitable distance to see
the surrounding airspace and react promptly if required. However,
if the remote pilot is required to perform the airspace scanning,
the management of the flight must be such that it does not require
too much attention. For this reason, the requirement to have a pre-programmed
trajectory for the UA is established when operating without VOs. However,
the remote pilot must always have the ability to intervene in order
to manoeuvre the UA (e.g. for collision avoidance).
\item If VOs are used, the UAS operator is required to ensure that: 
\begin{enumerate}
\item the VOs are positioned so that they can provide adequate coverage
of the operational volume and the surrounding airspace with the minimum
flight visibility indicated, and there are no potential terrain obstructions;
\item the distance between any VO and the remote pilot is not more than
1 km, to ensure better control of VOs and their communication with
the remote pilot; 
\item robust and effective communication means are available for the communication
between the remote pilot and the VOs;
\item if means are used by the VOs to determine the position of the UA,
those means are functioning and effective; and 
\item the VOs have been briefed on the intended path of the UA and the associated
timing
\end{enumerate}
\end{enumerate}
Rules on remote pilot competency 

Contingency and emergency procedures to be defined in OM by operator

Emergency response plan must be devised 

Technical requirements in STS-02:

\section{Indian efforts}

\subsection{@Dunzo consortium}

\section{Use cases}

\begin{lyxgreyedout}
@Include Apurva/Swaroop's comparison table%
\end{lyxgreyedout}


\section{Comparison of Scenarios}

\section{Comparison of Use cases}

\section{Applications by Industry}

\subsection{Agriculture}

\begin{enumerate}

   \item Seeding/aforestation

\begin{enumerate}
   \item VLOS/BVLOS(?) STS01/02(?)
   \item Uncontrolled
   \item Remote ID not necessary
   \item Risks: ??
\end{enumerate}
   \item Spraying
\begin{enumerate}
         \item Payload 
         \item VLOS STS01/02
         \item Uncontrolled
         \item Remote ID not necessary
         \item Risks: Chemical / Chemical OOP / Water bodies / Rain
\end{enumerate}
         \item Surveying / Fixed wing
\begin{enumerate}
  \item VLOS/BVLOS STS01/02
  \item Uncontrolled
  \item Remote ID
  \item Risks: Altitude, range
\end{enumerate}

\end{enumerate}

\subsection{Survey} 

\begin{enumerate}


\item Land use
\begin{enumerate}
  \item Fixed wing long range survey: 
    Risks: High Collisions, and higher altitude
\end{enumerate}

\item Infrastructure
Risks common: Higher collision probability 
\begin{enumerate}
  \item Railways \\
    Risks: EMI/EMC
  \item Roadways \\
    ??
  \item Power distribution \& related \\
    Risks: EMI/EMC
  \item Cell tower \\
    Risks: EMI/EMC
  \item Industrial plants \\
    Risks of exterior or interior surveys: Chemical Hazards, Low visibility
  \item Dams/Bridges \\
    UC: find visible damage
    Risks: ??
  \item Solar Panel \\
    UC: damaged cells \\
    UC: cleaning \\
    Risks: ??
  \item Wind farms \\
    Risks: EMI/EMC
 
\end{enumerate}
\end{enumerate}

\begin{verbatim}
Generic risks:
    Mishap: Payload delivered to wrong hands
    Mishap: Payload intercepted by misbehaving actor

Emergency Use cases
  Priority: Higher

- Medical: First Responder/Emergency Response
  - Equipment: sending payload
      Payload: Devices: Defibrillator/AED, Resuscitation, Mobile device for 
        telemedicine
      Operational specifics: BVLOS, Day/Night, One-way mission
      Risks: Higher cost payload, Higher velocity

  - Medical aid/Drug delivery
      Payload: first responder medical kits, etc.
      Operational specifics: BVLOS, Day/Night, One-way
      Risks: Chemical, Higher cost payload, Higher velocity

  - Blood/Organ delivery
      Payload: Blood, Organs
      Operational specifics: BVLOS, Day/Night, One-way
      Risks: Biohazard, Higher cost payload, Higher velocity

  - Pathology sample collection
      Payload: Blood/Tissue/etc.
      Operational specifics: BVLOS, Day/Night, One-way
      Risks: Biohazard, Higher velocity
    
- Disaster
  - Survey
      Payload: Regular/Thermal/?? Camera
      Operational specifics: 
      Risks:

  - Recovery
      Operational specifics: 
      Risks:

- Medical non-Emergency
  - Surgical equipment & OT materials (??) delivery
      Payload: sterilised and unsterilised equipment
      Operational specifics: 
      Risks: Higher cost payload

Entertainment

1.  Drone light show
    Payload: LED array
    Operational Specifics: VLoS
    Risk factor: Swarm, Loss of automated Control requires manual takeover 
      (higher risk because non-trivial operation), OOP
    Examples: EHang, Intel, ETH-Zurich spinoff, UGCS (GCS for swarm)

Aerial Photo-/video-graphy/Photogrammetry

1.  Filming
    Payload: Camera, Light meters
    Operational Specifics: VLOS, OOP, (maybe) Constrained environments,
    Risks: High cost equipment as payload

1.  Survey Climbing/Mountaineering
    Payload: Camera, Light meters
    Operational Specifics: VLOS
    Risks: endangering ecological balance (avalanche, etc.), high wind, rain,
       snow, low temperatures
    
1.  Monitor/Tracking Wildlife
    Payload: Camera, Light meters
    Operational Specifics: VLOS
    Risks: Protected sites, endangering wildlife, ecosystem and habitat, 
      infrared lighting for night  vision videography

1.  Monuments
    Payload: Camera, Light meters, 
    Operational Specifics: VLOS, High cost equipment as payload 
    Risks: Protected sites, etc.

1.  LIDAR survey
    Payload: LIDAR
    Operational Specifics: BVLoS, VLOS, High cost equipment as payload
    Risks: High risk environments (mines, etc.), constrained environment, 
      (maybe) protected architectural sites, etc., inflammable materials?

Law Enforcement

1.  Surveillance
    Payload: Camera
    Operational Specifics: BVLoS, VLOS, High cost equipment as payload, 
      Sensitive information on board
    Risks: OOP, Operation in varied environments (dynamic)

1.  Drone-based Counter UAS
    Payload: Camera, varied Takedown mechanisms
    Operational Specifics: VLOS
    Risks: Assume all risks of the targeted operation; damage to property 
      and human life; depends on takedown mechanism

TCL ?? Non-drone based Counter UAS
    Payload: Camera, varied Takedown mechanisms
    Risks: 

\end{verbatim}


\subsection{Disaster}
\begin{enumerate}
  \item Measuring levels in water bodies for flooding
  \item Radioactivity measurement near nuclear power plants
\end{enumerate}


\cleardoublepage{}

\chapter{Risk Assessment}

\section{Methodology}

We begin with the identification of hazards, risks and threats to
a UAS operation. Some of the terminology used this point onward will
have domain specific interpretation as noted in 

\section{Hazard/Risk/Threat Identification}

\subsection{Ground Risk Params}

\subsubsection{People}
\begin{enumerate}
\item Population-related
\begin{enumerate}
\item Permanently or cyclic populated areas (cycle < 1 day)
\item Dense areas (city centre streets, etc.)
\item Sensitive areas (schools, hospitals, etc.)
\item Occasional or seasonal events (concerts, stadiums, beaches, etc.)
\end{enumerate}
\item Security-related
\begin{enumerate}
\item Military barracks
\item Summits
\item VIP protection
\end{enumerate}
\end{enumerate}

\subsubsection{Infrastructure}
\begin{enumerate}
\item Industry-related
\begin{enumerate}
\item Permanent and non-permanent industrial sites
\item Chemical and Nuclear sites
\item Laboratories
\item Wind farms, power stations
\item Cranes
\end{enumerate}
\item Transport-related
\begin{enumerate}
\item Airports, aerodromes and identified take-off and landing sites, model-flying
sites
\item Roads and highways
\item Harbours
\item Rail
\end{enumerate}
\item Security-related
\begin{enumerate}
\item Military areas
\end{enumerate}
\end{enumerate}

\subsubsection{Environment}
\begin{enumerate}
\item Animal Reservations
\end{enumerate}

\subsection{Air Risk Params}
\begin{enumerate}
\item High probability of traffic (hospitals, etc.) 
\item Seasonal or permanent recreational activities (base jump, flying suits,
kite surf, etc.) 
\item Localised events (hotels water jets, geysers, etc.)
\item Airports, aerodromes and identified take-off and landing sites, model-flying
sites
\end{enumerate}

\subsection{Payload Risk Params}
Bio hazard: (BSL1- BSL 4) https://www.who.int/csr/resources/publications/biosafety/Biosafety7.pdf?ua=1 \\
Chemical hazard (cat1- cat5) https://www.osha.gov/dsg/hazcom/ghsguideoct05.pdf


\begin{table}[tbh]
\small
\ttfamily
\begin{centering}
\begin{longtable}{|l|l|l|l|}
 \hline
 \multicolumn{4}{|c|}{Risk/Hazard/Threat Parameter Categorisation Level} \\
 \hline
 1 & 2 & 3 & 4 \\
 \hline
 \multirow{21}{3em}{Ground} & \multirow{8}{7em}{People}  & \multirow{4}{5em}{Population} & Permanently or cyclic \\
  & & & populated areas \\\cline{4-4}
  & & & Dense areas \\\cline{4-4}
  & & & Sensitive areas \\\cline{4-4}
  & & & Occasional or seasonal events \\\cline{3-4}
  & & \multirow{3}{5em}{Security} & Military barracks \\
  & & & Summits \\
  & & & VIP protection \\\cline{2-4}
  & \multirow{12}{5em}{Infrastructure} & \multirow{5}{3em}{Industry} & Permanent and non-permanent \\
  & & & industrial sites \\\cline{4-4}
  & & & Chemical and Nuclear sites \\\cline{4-4}
  & & & Laboratories \\\cline{4-4}
  & & & Wind farms, power stations \\\cline{4-4}
  & & & Cranes \\\cline{3-4}
  & & \multirow{6}{5em}{Transport} & Airports, aerodromes and identified \\
  & & & take-off and landing sites, \\
  & & & model-flying sites \\\cline{4-4}
  & & & Roads and highways \\\cline{4-4}
  & & & Harbours \\\cline{4-4}
  & & & Rail \\\cline{3-4}
  & & Security & Military areas \\\cline{3-4}
  & & Environmental & Animal Reservations \\\cline{1-4}
 \multirow{7}{3em}{Air} & & & High probability of traffic \\\cline{4-4}
  & & & Seasonal or permanent recreational \\
  & & & activities \\\cline{4-4}
  & & & Localised events \\\cline{4-4}
  & & & Airports, aerodromes and identified  \\
  & & & take-off and landing sites,  \\
  & & & model-flying sites \\

\hline
\end{longtable}
\end{centering}
\end{table}


\subsection{Other Threats}
\begin{enumerate}
\item Electro-magnetic wave-emitting sites (radars, high-voltage lines,
solar farms, etc.) 
\item GNSS-outage forecast areas
\end{enumerate}

\section{Mitigations}

TBD

\section{SORA}

\paragraph{When can SORA be applied}

For the ground risk, ground risk classes are assigned and barriers
that can mitigate the death and destruction on the ground are identified.
The final lethality of the ground risk can then be determined. If
the operation in question is BVLOS over a populated area with a UA
of 3m or more, or VLOS over a populated area with a UA of 8m or more,
the ground risk is so great that the SORA is not an appropriate tool
for ensuring safety

\paragraph{Airspace Encounter Category}

The perceived level of air risk \textendash{} the risk of a mid-air
collision - is incorporated through an Airspace Encounter Category
(AEC) for a given region of airspace. The SORA method assigns an AirRisk
Class (ARC) ranging from 1 (low risk) to 4 (high risk) to these AECs
\textendash{} see Table below \textendash{} based on three factors:
the rate of proximity, dependent on the number of aircraft assumed
to be in the airspace; the geometry of the aircraft, use of specific
routes etc., in the airspace; dynamics, or how fast aircraft travel
in the airspace. Measures can be proposed to reduce these impacts.

\paragraph{Enhancement in EU in SORA}

The draft EU implementing and delegated acts envisage geo-awareness
and other features for the design of certain drones. Geo-awareness
gives rise to the ability to create \textquotedblleft geo-fences\textquotedblright{}
that a drone must stay one side of, of through which their passage
may be controlled. Geo-fencing applications are becoming available
that are capable of preventing drones from approaching restricted
areas, such as airports, or on the contrary, ensuring that they do
not fly outside of a given authorised area. Such systems could be
used to restrict access to certain areas of sensitive airspace (geo-restriction)
or to create \textquotedblleft dronodromes\textquotedblright{} where,
for example, novice open category users could play without interfering
with other airspace users (geo-caging). Deciding which of these will
apply to a given airspace zone is one of the major tasks of airspace
assessment. As this technology improves, with quantitative measures
and precision methods based on reliable error margins and manned aviation
data, errors can be reduced and safe operational areas can be mapped.
However, these areas should not be fixed. Based on these technological
advances, the relevant authority could decide to re-define security
limits, allowing drones to operate closer to airports for example.

\paragraph{Dynamic remoteID/operationID generation based on operational space}

\section{Indian efforts}

\subsection{@Dunzo consortium}

\section{Comparison of Strategies}

\chapter{Appendix}

\section{Pending changes}

\begin{enumerate}

	\item Integrate Apurva's write up on
	\begin{enumerate}
		\item Hazards and risks for urban flight (as per NASA recommendations)
		\item List of common hazards (as per FAA Order)
	\end{enumerate}
from \href{doc}{https://docs.google.com/document/d/1eHDIJ4yoV9JgGWHc02FygwE4SounAoLkCo1Ykyhjjro/edit}
\item Merge NASA and FAA material in our Risk Assessment paper due to the major overlap in material

\end{enumerate}

\bibliographystyle{IEEEtran}
\bibliography{IEEEabrv,IEEEexample,../Bibliography}

\end{document}
